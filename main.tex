\documentclass[12pt]{report}

%====================== PACKAGES ======================


\usepackage{pgfgantt}

\usepackage[french]{babel}
\usepackage{tikz-uml}
\usepackage[french]{translator}
\usetikzlibrary{babel}
\usepackage[utf8x]{inputenc}
\usepackage[T1]{fontenc}
%pour gérer les positionnement d'images
\usepackage{float}
\usepackage{amsmath}
\usepackage{graphicx}

\usepackage{url}
%pour les informations sur un document compilé en PDF et les liens externes / internes
\usepackage{hyperref}
%pour la mise en page des tableaux
\usepackage{array}
\usepackage{tabularx}
%pour utiliser \floatbarrier
%\usepackage{placeins}
%\usepackage{floatrow}
%espacement entre les lignes
\usepackage{setspace}
%modifier la mise en page de l'abstract
\usepackage{abstract}
%police et mise en page (marges) du document
\usepackage[T1]{fontenc}
\usepackage[top=2cm, bottom=2cm, left=2cm, right=3.2cm]{geometry}
%Pour les galerie d'images
\usepackage{subfig}
\tikzumlset{fill class=orange!5, fill component=white, fill usecase=black!25, fill object=orange!25}
%====================== INFORMATION ET REGLES ======================

%rajouter les numérotation pour les \paragraphe et \subparagraphe
\setcounter{secnumdepth}{4}
\setcounter{tocdepth}{4}



%======================== DEBUT DU DOCUMENT ========================

\begin{document}

%régler l'espacement entre les lignes
\newcommand{\HRule}{\rule{\linewidth}{0.5mm}}

%page de garde
\begin{titlepage}
\begin{center}

% Upper part of the page. The '~' is needed because only works if a paragraph has started.
\includegraphics[width=0.35\textwidth]{./logo}~\\[1cm]

%\textsc{\LARGE Université ou Entreprise}\\[1.5cm]

\textsc{\Large }\\[0.5cm]

% Title
\HRule \\[0.4cm]


{\huge \bfseries Découverte du framwork Apache OFBiz.\\
	Dévéloppement d'une API HTTP, basé sur le style architectural REST et inégration dans un contexte de projet client. \\[0.4cm] }

\HRule \\[1.5cm]


% Author and supervisor
\begin{minipage}{0.4\textwidth}
\begin{flushleft} \large
\emph{Auteur:}\\
Artemiy \textsc{Rozovyk}\\
\thispagestyle{empty}
\end{flushleft}
\end{minipage}
\begin{minipage}{0.4\textwidth}
\thispagestyle{empty}
\begin{flushright} \large
\emph{Tuteur de stage:} \\
Mathieu \textsc{Lirzin}\\
\emph{Référent:} \\
Florent \textsc{Foucaud}
\end{flushright}
\end{minipage}

\vfill

% Bottom of the page
{\large \today}

\end{center}
\end{titlepage}
%ne pas numéroter cette page
\newpage
\tableofcontents
\newpage


\renewcommand{\abstractnamefont}{\normalfont\Large\bfseries}
%\renewcommand{\abstracttextfont}{\normalfont\Huge}

\renewcommand{\abstractname}{Remerciements}
\addcontentsline{toc}{chapter}{Remerciements}
\begin{abstract}
		Merci tout le monde!
\end{abstract}

\newpage




%espacement entre les lignes d'un tableau
\renewcommand{\arraystretch}{1.5}

%====================== INCLUSION DES PARTIES ======================


%recommencer la numérotation des pages à "1"


\chapter{Introduction}

Le présent document expose le travail effectué lors de mon stage de fin de licence au sein de la société Néréide.
Ce stage se décompose en deux parties.
La première partie a pour but de se familiariser avec le framework
orienté \emph{progiciel de gestion intégré}
Apache OFBiz et avec son utilisation dans le contexte de la société d'accueil. 
La deuxième partie consiste à intégrer un système permettant la définition des API HTTP
du style REST,
 ainsi que la modification d'une API existante afin de donner une preuve de concept. \\
Le travail a été effectué en étroite collaboration avec le principal concerné - la 
communauté Apache,
ce qui a contribué à une meilleure cohérence entre le travail réalisé et les besoins des utilisateurs.\\
Dans un premier temps nous allons présenter l'entreprise d'accueil ainsi que faire une description de
l'outil principal utilisé.
Dans un deuxième temps nous exposerons la démarche qui a permis une compréhension suffisante du framework OFBiz
 nécessaire à la partie finale du stage, REST, qui sera décrite dans un troisième temps.



\chapter{Contexte du stage}

Intro

\section{Entreprise}

\subsection{Presentation générale }
Néréide est une société de services en logiciels libres crée en 2004  SCOP SARL \\
parler de méthodes agiles. 
transparence interne totale ;
fonctionnement démocratique (SCOP, co-gérance tournante);
implication des salariés dans tous les domaines de décision ;
salaire unique


























\subsection{Domaine}
Retail
\subsection{Activité}
\label{activite}
Transparence libre entreprise \\


Integration(utilisation des briques en tant que telles) et
Dévéloppement spécifique (adaptation d'OFBIZ pour les besoins ) plugins propres à la logique métier. Pas de forfait(contrat doit etre rempli, l'entreprise s'engage à ce qu'il soit livré dans les tepms), mais la régie (sont payé au temps de travail et pas à condition de remplir un ... ).  Pareil avec Décath 6 pers font de la régie...
Administration système
\subsection{Projets}
Décathlon
 rfid store, interface pour les achteurs et producteurs, dev front pour le store specifique au lieu d'écrans OFBiz, communication avec des API...

Dejbox
\\
Dejbox est une société de la foodTech qui propose aux salariés d’entreprise de leur livrer des repas directement sur leur lieu de travail. L’ensemble des vente est réalisé au travers d’un site e-commerce par lequel le salarié commande un repas.

Le projet a pour objectif de mettre en œuvre un outil de type ERP afin de gérer la chaîne de réapprovisionnement en produit frais vendu en ligne. Il s’agit donc de créer un référentiel d’article et de fournisseur et de pouvoir saisir des commandes qui seront envoyées aux fournisseurs et réceptionnées suite à leur livraison. Enfin, il s’agit de mettre en place la sortie de stock en intégrant les consommations de produit provenant du site de vente en ligne.


....
Dépuis 2013 les dev remontantes à la communauté , parce que besoin de support,  divergence. moutons-acteur 






\newpage
\section{Framework OFBiz}
\subsection{Vue d'ensemble }
\emph{Open For Business (OFBiz)} est une suite d'applications pour la gestion de l'entreprise qui se base sur une architecture très couramment utilisé \emph{(MVC)} et qui implémente des composantes classiques de gestion des donnés, de logique métier, et de traitement spécialisé. 

On peut notamment distinguer les modules génériques destinées à la gestion des tâches communes à la plupart des entreprises, telles que la gestion des stocks, la comptabilité, la facturation et bien d'autres. Quant à leur structure, toutes les composantes sont étroitement liées entres elles, ce qui facilite la compréhension, l'utilisation et la personnalisation de ces dernières. 


En plus d'une architecture qui encourage la customisation, OFBiz est entièrement distribué en tant que \emph{open source software}\footnote{Logiciel libre sous licence \href{https://www.apache.org/licenses/LICENSE-2.0.html}{ASL2 (Apache License Version 2.0)} ce qui donne le droit de personnaliser, d'étendre, de restructurer et de vendre le système concerné. } ce qui le rend particulièrement intéressant car le logiciel développé à base d'OFBiz n'est pas soumis à la condition d'être libre comme c'est le cas de la licence GPL  \footnote{\href{http://www.gnu.org/licenses/gpl-3.0.html}{GNU General Public Licence}} par exemple.

\subsection{Architecture }
D'un point de vue purement technique OFBiz se base sur la plateforme Java ainsi que sur l'utilisation des DSL\footnote{Domaine specific language \emph{(Langages spé-cifiques au domaine)}} basés sur des grammaires écrites en XML (mettre la bib). En ce qui concerne la partie principale du framework, les échanges HTTP sont implémentes par une extension de la classe \verb=HTTPServlet= \cite{chan2017servlet} et la communication avec les bases de données se fait via l'API Java JDBC \cite{JDBC}.

Dans sa structure on distingue \emph{le framework}, \emph{les applications} et \emph{les plugins}. Le \emph{framework} comporte l'ensemble des outils et des mécanismes techniques, notamment en matière de  communication réseau et d'interaction entre les différentes applications.
Les principaux composants métier tels que la comptabilité, la gestion des stock, ou la facturation se trouvent dans la partie \emph{applications}. 
Finalement la notion du plugin correspond à une application spécifique qui repose sur des composantes générales: par exemple le plugin \emph{eCommerce} correspond à un boutique en ligne qui utilise des nombreuses  \emph{applications} comme \emph{la gestion du stock} ou \emph{la facturation}. 

\subsection{DSL XML}
L'une des particularités d'OFBiz ce sont des fichiers XML qui servent à déclarer entre autres
des routes HTTP, des pages de rendu appelés \emph{Écrans}, ainsi que des services. Le principe est de transformer des informations sous format XML facilement compréhensibles par le développeur, en objets Java correspondants. 


\definecolor{dollarbill}{rgb}{0.52, 0.73, 0.4}
\subsection{Container}
L'interface container représenté sur la figure \ref{container} permet de définir des objets qui correspondent à des processus qui peuvent être initialisés, démarrés et arrêtés. L'intérêt est de pouvoir lancer un daemon spécifique en parallèle de l'execution d'OFBiz comme c'est le cas de \verb|EntityDataLoadContainer| qui est responsable du chargement des données et leur mise à jour en cas de modification du modèle. Quand à \verb|TestRunContainer| il s'assure du lancement des testes unitaires grâce à un mécanisme spécifique du framework. 
\begin{figure}
	\centering
	\begin{tikzpicture}
	\umlinterface[scale=0.9,x=0,y=5]{Container}{}{
		\umlvirt{+ init(cmds : List<StartupCommand>, name : String,} \\
		\umlvirt{\hspace{1,1cm}config : String) : void} \\
		\umlvirt{+ start() : void} \\
		\umlvirt{+ stop() : void} \\
		\umlvirt{+ getName() : String}
	}
	\umlclass[ x=-5, y=-2]{EntityDataLoadContainer}{
		- name : String \\
		- module :String
	}{
		\umlvirt{+ getName() : String} \\
		\umlvirt{+ createDbConstraints(...) : void}\\
		\umlvirt{+ dropPrimaryKeys(...) : void}
	}
	\umlclass[ x=6, y=-2]{TestRunContainer}{
		- name : String \\
		- jsWrapper : JunitSuiteWrapper
	}{
		\umlvirt{+ getName() : String} \\
		\umlvirt{+ createJunitXmlListener(...) : JunitXmlListener} \\
		\umlvirt{+ logTestSuiteResults(...) : void} \\
	}

 \umldep[geometry=|-|, pos1=1.5, pos2=0.2 ,draw=dollarbill,  thick]{TestRunContainer}{Container}
 \umldep[geometry=|-|, pos1=1.5, pos2=0.2 ,draw=dollarbill,  thick]{EntityDataLoadContainer}{Container}
	\end{tikzpicture}
	
	
	\caption{Définition du type container}
	\label{container}
\end{figure}
\subsection{Composants}
Les éléments constitutifs de OFBIZ sont des composants. Un composant est un regroupement des containers, des entités, des services, des vues (\emph{Écrans}) et des applications Web.


L'exemple classique d'un composant est celui de \verb|webtools| qui assure la gestion technique de l'ensemble du système par l'administrateur via une application web, ce qui implique le fait que ce composant regroupe la plupart des éléments majeurs du framework.
Nous en tant que développeurs avons la possibilité de définir nos propres composants, notamment des \emph{plugins}. 
  


\subsection{Web applications}
\subsection{Entity engine}
\subsection{Service engine}
\subsection{Screen engine}
\subsection{Fonctionnel métier}

\newpage
\section{Sujet de stage }



\subsection{API REST au sein d'OFBiz}

Jira

 
\chapter{Travail réalisé}


\section{Aperçu général}
Voici la chronologie du travail réalisé en entreprise.\\
\ganttset{%
	calendar week text={%
		\pgfcalendarmonthshortname{\startmonth}~\startday%
	}%
}
\newganttlinktype{f-m}{
	\ganttsetstartanchor{on right=1}
	\ganttsetendanchor{on left=0}
	\draw[/pgfgantt/link]
	([xshift=-.2pt]\xLeft, \yUpper) --       % xshift to fit arrow
	node[pos=.5, /pgfgantt/link label node] {\ganttlinklabel} 
	(\xRight, \yLower);
}


%vgrid={*1{blue!30},
%	*6{black,dotted},
%	*1{red!30},
%	*2{black,dotted},
%	*1{blue!30},
%	*{34}{black,dotted},
%	*1{green!30},
%	*1{red!30},
%	*{10}{black,dotted},
%	*1{green!30}},
\setganttlinklabel{f-m}{}

\begin{ganttchart}[
	hgrid={*1{black!30,dotted}},
	vgrid={*1{black!30,dotted}},
	x unit=3mm,
	time slot format=isodate,
	inline,
	bar/.append style={fill=blue!37},
	bar height=.5,
	group/.append style={draw=black, fill=black!50},
	milestone/.append style={fill=green!20, rounded corners=6pt,scale=2},
	milestone inline label node/.append style={right=1mm},
	]{2019-03-28}{2019-05-25}
	\gantttitlecalendar{year, month=name, week} \\
	\ganttgroup{Analyse des besoins}{2019-03-29}{2019-04-9}\\
	\ganttgroup{Réalisation technique}{2019-04-5}{2019-05-13}\\
	\ganttgroup{Maintenance}{2019-05-13}{2019-05-24} \\
	\ganttbar[bar height=.5]{OFBiz}{2019-04-01}{2019-04-11}\\
	\ganttbar[
	bar/.append style={ fill=red!50
	}]{REST}{2019-04-08}{2019-04-28} \\
	\ganttbar[
	bar/.append style={ fill=orange!50
	}]{Entitymaint}{2019-04-20}{2019-05-12} \\

	\ganttmilestone{Preuve de concept}{2019-05-12}] \\
	\ganttbar[
	bar/.append style={ fill=purple!40, dashed
	}]{Revue de code}{2019-05-13}{2019-05-23} \\
	\ganttlink{elem3}{elem4}
	\ganttlink{elem4}{elem5}
	\ganttlink[link type=f-m]{elem5}{elem6}
	\ganttlink[link type=dr]{elem4}{elem6}
	\ganttlink[link type=f-m]{elem6}{elem7}
\end{ganttchart}

\newpage









\section{Environnement}

\subsection{Installation de l'environnement}
Avant tout, mon intégration a commencé par l'installation du poste de travail suivi par une discussion sur le choix de distribution Linux \footnote{Une question qui a, apparemment, beaucoup d'importance.} , la configuration des outils utilisés par l'entreprise ainsi que par la mise en place des accès aux ressources internes \footnote{Contenaient, à mon avis, des information sensibles, mais cela s'explique par le principe de transparence \ref{activite} }
IntelliJ qui facilite la navigation vu la specificté d'OFBIZ XML/JAVA




\subsection{Formation générale}
Découverte du fonctionnel basique, aperçu des techniques utilisés dans l'OFBIZ

\subsection{Jira}
Mécanisme de tickets pour les projets client, utilisé aussi par la communauté Apache...
\subsection{Approfondissement de Git }
dev à plusieurs
git-branching.js

\subsection{Découverte de communauté libre Apache}
 OFbiz est maintenu par des contributeurs commiteurs et PMC (what is), indépendants, les entreprise n'ont pas de pouvoir ni d'obligation de contribuer (mais elles ont l'interet), à vocation personnelle. Licence Apache et GPL (oblig legale de distribuer en libre)





\newpage
\section{Prise en main d'OFBiz}

\subsection{Premier plugin}

\subsection{Projets existants et leur structure}
\subsubsection{Décathlon}
RFID et tout ça
\subsubsection{Dejbox}
Pierre et Antoine ont tout géré 

\subsection{Problématique vis-à-vis du développement}
What is "fonctionnel"

\newpage

\section{Analyse de l'existant}
\subsection{ControlServlet}
\subsection{Mécanisme de résolution des URI}
\subsection{Filtres}
Delegateur et Dispatcher


\newpage

\section{Analyse des besoins et attentes de la maîtrise d'ouvrage}
\subsection{Structure générale des application web}
Les enjeux, les problématiques les solutions, *COURS MAURIZIO*
\subsection{API en cours}
RPC
\subsection{Controleur}
<request-map>...
\subsection{Besoins d'évolution}
Avenir
*Discussion communautaire*
\subsection{Representational state transfer}
\subsubsection{Histoire}
Roy Fielding
\subsubsection{Principe}
*Détailles du cours de Maurizio: idempotence, navigabilité par hyperlink, 
notion de ressource etc.
\subsubsection{Avantages}
\subsubsection{Examples d'API du style REST}
API REST de Twitter, SoundCloud, Wiktionnaire,\\
les différences entre la définition de Roy Fielding et l'implémentation de ces dernières

\subsection{Implementations existantes}
\subsubsection{Camel}
\subsubsection{JAX-RS}
Tentative d'intégration ---\\
ServletJaxRS fonctionnelle\\
Particularités techniques (annotations) \\
Conflit politique car n'est pas dans le même esprit de l'existant.\\



\newpage

\section{Réalisations techniques}

\subsection{Librairie CXF}
Problèmatique avec les dépendances supplementaires: 
Tika contient déjà le CXF
\subsection{Choix vers URITemplate}
description de classe
\subsection{\textit{OverrideView()} et le conflit avec les URI segmentées}
\subsection{Choix d'intégration en parallèle avec le système existant }
\subsection{Nouveau contrôleur}
\subsubsection{Compromis pour les conflits d'URI}

\subsection{Modification de la partie "Administration: gestion des entités"  (entitymaint)  }
\subsubsection{Choix de la partie illustrative}
\subsubsection{PUT vs POST}
\subsubsection{Clés composées}
\subsubsection{Formulaires génériques }
Create update dans un même formulaire.
\subsection{Stateless}
\subsubsection{Les réalisation par la communauté}

Jaques Le Roux Token en gardant la session.
\subsection{RESTClient pour la communauté}
\subsubsection{Généralisation de code}
\subsubsection{Correction d'incohérences}



\iffalse
\section{Besoins fonctionnels}

Après une analyse des besoins fonctionnels du projet, nous avons défini deux sous catégories. D'un côté, les besoins [...], de l'autre, les besoins [...].

\subsection{Sous-partie 1}

Bla

\subsection{Sous-partie 2}

Bla

\newpage

\section{Besoins non-fonctionnels}

Comme précédemment, nous avons choisi de distinguer deux catégories pour les besoins non-fonctionnels. D'une part, nous avons les besoins non-fonctionnels pour les [...], et d'autre part ceux pour [...]. Nous avons aussi pris en compte les contraintes de développement, que nous détaillerons à la fin de cette partie.

\subsection{Sous-partie 1}

Bla\\

Aperçu du rendu souhaité :

\begin{figure}[!h]
\begin{center}
\includegraphics[height=10cm]{besoins/rendu}
\end{center}
\caption{Rendu attendu}
\end{figure}

\subsection{Sous-partie 2}

Bla

\newpage

\section{Développement}

Intro

\subsection{Tâches}

Bla\\


%tableau à taille fixée sur certaines colonnes (param sur la ligne \begin{tabularx}, voir wiki pour plus d'info sur la syntaxe
\begin{figure}[!h]
\begin{center}
\begin{tabularx}{17cm}{|c|p{6cm}|X|}
  \hline
  Priorité & Nom & Raison\\
  \hline
  1 & Tache 1 & Doit être vérifié en premier car sinon [...] \tabularnewline
  2 & Tache 2 & On doit pouvoir [...] \tabularnewline
  3 & Tache 3 & Comme les principales fonctionnalités permettant de tester sont opérationnelles, nous pouvons passer à cette tâche. \tabularnewline
  4 & Tache 4 & Parce que [...] \tabularnewline
  5 & Tache 5 & La tache 5 fait partie des principales [...]. \tabularnewline
  6 & Tache 6 & Dernière fonctionnalité essentielle à mettre en place. \tabularnewline
  7 & Tache 7 & Non-essentiel, mais apporterait un plus au projet. \tabularnewline
  8 & Tache 8 & Non-essentiel, mais apporterait un plus au projet. \tabularnewline
  \hline
\end{tabularx}
\end{center}
\caption{Tableau récapitulatif des tâches}
\end{figure}

\subsection{Tests}

Bla\\

\begin{figure}[!h]
\begin{center}
\begin{tabularx}{17cm}{|p{6cm}|X|}
  \hline
  Fonctionnalité & Test\\
  \hline
  Fonction 1 & Quand [...], vérifier [...]. \tabularnewline
  & Et quand [...], vérifier [...]. \tabularnewline
  Fonction 2 & Vérifier [...]. \tabularnewline
  Fonction 3 & Vérifier [...]. \tabularnewline
  Fonction 4 & Avoir [...]. \tabularnewline
  Fonction 5 & Accéder à [...]. \tabularnewline
   & Vérifier que [...]. \tabularnewline
  Fonction 6 & Accéder à [...]. \tabularnewline
   & Et vérifier [...]. \tabularnewline
  Fonction 7 & Installer [...]. \tabularnewline
   & Vérifier [...]. \tabularnewline
  Fonction 8 & Compter [...]. \tabularnewline
  \hline
\end{tabularx}
\end{center}
\caption{Tableau récapitulatif des tests}
\end{figure}
\fi

\chapter{Conclusion}




\section{Lien avec les connaissences obtenu lors de la formation universitaire}
METTRE DANS LA CONCLUSION
\subsection{MVC}
\subsection{Servlet}
\subsection{raeeMarker -JSP}
\subsection{Notion d'entié - Symony}
\subsection{Routage}
\subsection{Gradle Maven}
\subsection{Testes Unitaires}
L'utilité des testes : surtout démontrer l'utilisation 


\newpage

%récupérer les citation avec "/footnotemark"
\nocite{*}

%choix du style de la biblio
\bibliographystyle{plain}
%inclusion de la biblio
\bibliography{bibliographie.bib}
%voir wiki pour plus d'information sur la syntaxe des entrées d'une bibliographie

\end{document}