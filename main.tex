\documentclass[a4paper]{report}

%====================== PACKAGES ======================


\usepackage{pgfgantt}

\usepackage[frenchb]{babel}
\usepackage[frenchb]{translator}
\usetikzlibrary{babel}
\usepackage[utf8x]{inputenc}
%pour gérer les positionnement d'images
\usepackage{float}
\usepackage{amsmath}
\usepackage{graphicx}

\usepackage{url}
%pour les informations sur un document compilé en PDF et les liens externes / internes
\usepackage{hyperref}
%pour la mise en page des tableaux
\usepackage{array}
\usepackage{tabularx}
%pour utiliser \floatbarrier
%\usepackage{placeins}
%\usepackage{floatrow}
%espacement entre les lignes
\usepackage{setspace}
%modifier la mise en page de l'abstract
\usepackage{abstract}
%police et mise en page (marges) du document
\usepackage[T1]{fontenc}
\usepackage[top=2cm, bottom=2cm, left=2cm, right=2cm]{geometry}
%Pour les galerie d'images
\usepackage{subfig}

%====================== INFORMATION ET REGLES ======================

%rajouter les numérotation pour les \paragraphe et \subparagraphe
\setcounter{secnumdepth}{4}
\setcounter{tocdepth}{4}



%======================== DEBUT DU DOCUMENT ========================

\begin{document}

%régler l'espacement entre les lignes
\newcommand{\HRule}{\rule{\linewidth}{0.5mm}}

%page de garde
\begin{titlepage}
\begin{center}

% Upper part of the page. The '~' is needed because only works if a paragraph has started.
\includegraphics[width=0.35\textwidth]{./logo}~\\[1cm]

%\textsc{\LARGE Université ou Entreprise}\\[1.5cm]

\textsc{\Large }\\[0.5cm]

% Title
\HRule \\[0.4cm]


{\huge \bfseries Découverte du framwork Apache OFBiz.\\
	Dévéloppement d'une API HTTP, basé sur le style architectural REST et inégration dans un contexte de projet client. \\[0.4cm] }

\HRule \\[1.5cm]


% Author and supervisor
\begin{minipage}{0.4\textwidth}
\begin{flushleft} \large
\emph{Auteur:}\\
Artemiy \textsc{Rozovyk}\\
\thispagestyle{empty}
\end{flushleft}
\end{minipage}
\begin{minipage}{0.4\textwidth}
\thispagestyle{empty}
\begin{flushright} \large
\emph{Tuteur de stage:} \\
Mathieu \textsc{Lirzin}\\
\emph{Référent:} \\
Florent \textsc{Foucaud}
\end{flushright}
\end{minipage}

\vfill

% Bottom of the page
{\large \today}

\end{center}
\end{titlepage}
%ne pas numéroter cette page
\newpage
\tableofcontents
\newpage


\renewcommand{\abstractnamefont}{\normalfont\Large\bfseries}
%\renewcommand{\abstracttextfont}{\normalfont\Huge}

\renewcommand{\abstractname}{Remerciements}
\addcontentsline{toc}{chapter}{Remerciements}
\begin{abstract}
		Merci tout le monde!
\end{abstract}

\newpage




%espacement entre les lignes d'un tableau
\renewcommand{\arraystretch}{1.5}

%====================== INCLUSION DES PARTIES ======================


%recommencer la numérotation des pages à "1"


\chapter{Introduction}

Le présent document expose le travail effectué lors de mon stage de fin de licence au sein de la société Néréide.
Ce stage se décompose en deux parties.
La première partie a pour but de se familiariser avec le framework
orienté \emph{progiciel de gestion intégré}
Apache OFBiz et avec son utilisation dans le contexte de la société d'accueil. 
La deuxième partie consiste à intégrer un système permettant la définition des API HTTP
du style REST,
 ainsi que la modification d'une API existante afin de donner une preuve de concept. \\
Le travail a été effectué en étroite collaboration avec le principal concerné - la 
communauté Apache,
ce qui a contribué à une meilleure cohérence entre le travail réalisé et les besoins des utilisateurs.\\
Dans un premier temps nous allons présenter l'entreprise d'accueil ainsi que faire une description de
l'outil principal utilisé.
Dans un deuxième temps nous exposerons la démarche qui a permis une compréhension suffisante du framework OFBiz
 nécessaire à la partie finale du stage, REST, qui sera décrite dans un troisième temps.



\chapter{Contexte du stage}

Intro

\section{Entreprise}

\subsection{Presentation générale }
Néréide est société de services en logiciels libres crée en 2004  


























\subsection{Domaine}
Retail
\subsection{Activité}
Dévéloppement spécifique 
Maintenance et support applicatif
Administration système
\subsection{Projets}
Decathlon
Dejbox








\section{L'architecture de OFBiz}
\subsection{Vue d'ensemble sur l'architecture}
DSL en XML, servlettes, jdbc ...
\subsection{Composants }

\subsection{Container}
\subsection{Web applications}
\subsection{Entity engine}
\subsection{Service engine}
\subsection{Screen engine}
\subsection{Fonctionnel métier}


\section{Sujet de stage }



\subsection{API REST au sein d'OFBiz}


 
\chapter{Travail réalisé}

\section{Environnement}

\subsection{Installation du l'environnement}
\subsection{Conventions}
\subsection{Formation développeur générale}
\subsection{Jira}
\subsection{Approfondissement de Git }
\subsection{Découverte de communauté libre Apache}

\section{Prise en main d'OFBiz}

\subsection{Premier plugin}

\subsection{Projets existants et leur structure}
\subsubsection{Décathlon}
RFID et tout ça
\subsubsection{Dejbox}
Pierre et Antoine ont tout géré 

\subsection{Problématique vis-à-vis du développement}
What is "fonctionnel"

\section{Analyse de l'existant}
\subsection{ControlServlet}
\subsection{Mécanisme de résolution des URI}
\subsection{Filtres}
Delegateur et Dispatcher


\section{Analyse des besoins et attentes de la maîtrise d'ouvrage}
\subsection{Structure générale des application web}
Les enjeux, les problématiques les solutions, *COURS MAURIZIO*
\subsection{API en cours}
RPC
\subsection{Controleur}
<request-map>...
\subsection{Besoins d'évolution}
Avenir
*Discussion communautaire*
\subsection{Representational state transfer}
\subsubsection{Histoire}
Roy Fielding
\subsubsection{Principe}
*Détailles du cours de Maurizio: idempotence, navigabilité par hyperlink, 
notion de ressource etc.
\subsubsection{Avantages}
\subsubsection{Examples d'API du style REST}
API REST de Twitter, SoundCloud, Wiktionnaire,\\
les différences entre la définition de Roy Fielding et l'implémentation de ces dernières

\subsection{Implementations existantes}
\subsubsection{Camel}
\subsubsection{JAX-RS}
Tentative d'intégration ---\\
ServletJaxRS fonctionnelle\\
Particularités techniques (annotations) \\
Conflit politique car n'est pas dans le même esprit de l'existant.\\





\section{Réalisations techniques}

\subsection{Librairie CXF}
Problèmatique avec les dépendances supplementaires: 
Tika contient déjà le CXF
\subsection{Choix vers URITemplate}
description de classe
\subsection{\textit{OverrideView()} et le conflit avec les URI segmentées}
\subsection{Choix d'intégration en parallèle avec le système existant }
\subsection{Nouveau contrôleur}
\subsubsection{Compromis pour les conflits d'URI}

\subsection{Modification de la partie "Administration: gestion des entités"  (entitymaint)  }
\subsubsection{Choix de la partie illustrative}
\subsubsection{PUT vs POST}
\subsubsection{Clés composées}
\subsubsection{Formulaires génériques }
Create update dans un même formulaire.
\subsection{Stateless}
\subsubsection{Les réalisation par la communauté}

Jaques Le Roux Token en gardant la session.
\subsection{RESTClient pour la communauté}
\subsubsection{Généralisation de code}
\subsubsection{Correction d'incohérences}



\iffalse
\section{Besoins fonctionnels}

Après une analyse des besoins fonctionnels du projet, nous avons défini deux sous catégories. D'un côté, les besoins [...], de l'autre, les besoins [...].

\subsection{Sous-partie 1}

Bla

\subsection{Sous-partie 2}

Bla

\newpage

\section{Besoins non-fonctionnels}

Comme précédemment, nous avons choisi de distinguer deux catégories pour les besoins non-fonctionnels. D'une part, nous avons les besoins non-fonctionnels pour les [...], et d'autre part ceux pour [...]. Nous avons aussi pris en compte les contraintes de développement, que nous détaillerons à la fin de cette partie.

\subsection{Sous-partie 1}

Bla\\

Aperçu du rendu souhaité :

\begin{figure}[!h]
\begin{center}
\includegraphics[height=10cm]{besoins/rendu}
\end{center}
\caption{Rendu attendu}
\end{figure}

\subsection{Sous-partie 2}

Bla

\newpage

\section{Développement}

Intro

\subsection{Tâches}

Bla\\


%tableau à taille fixée sur certaines colonnes (param sur la ligne \begin{tabularx}, voir wiki pour plus d'info sur la syntaxe
\begin{figure}[!h]
\begin{center}
\begin{tabularx}{17cm}{|c|p{6cm}|X|}
  \hline
  Priorité & Nom & Raison\\
  \hline
  1 & Tache 1 & Doit être vérifié en premier car sinon [...] \tabularnewline
  2 & Tache 2 & On doit pouvoir [...] \tabularnewline
  3 & Tache 3 & Comme les principales fonctionnalités permettant de tester sont opérationnelles, nous pouvons passer à cette tâche. \tabularnewline
  4 & Tache 4 & Parce que [...] \tabularnewline
  5 & Tache 5 & La tache 5 fait partie des principales [...]. \tabularnewline
  6 & Tache 6 & Dernière fonctionnalité essentielle à mettre en place. \tabularnewline
  7 & Tache 7 & Non-essentiel, mais apporterait un plus au projet. \tabularnewline
  8 & Tache 8 & Non-essentiel, mais apporterait un plus au projet. \tabularnewline
  \hline
\end{tabularx}
\end{center}
\caption{Tableau récapitulatif des tâches}
\end{figure}

\subsection{Tests}

Bla\\

\begin{figure}[!h]
\begin{center}
\begin{tabularx}{17cm}{|p{6cm}|X|}
  \hline
  Fonctionnalité & Test\\
  \hline
  Fonction 1 & Quand [...], vérifier [...]. \tabularnewline
  & Et quand [...], vérifier [...]. \tabularnewline
  Fonction 2 & Vérifier [...]. \tabularnewline
  Fonction 3 & Vérifier [...]. \tabularnewline
  Fonction 4 & Avoir [...]. \tabularnewline
  Fonction 5 & Accéder à [...]. \tabularnewline
   & Vérifier que [...]. \tabularnewline
  Fonction 6 & Accéder à [...]. \tabularnewline
   & Et vérifier [...]. \tabularnewline
  Fonction 7 & Installer [...]. \tabularnewline
   & Vérifier [...]. \tabularnewline
  Fonction 8 & Compter [...]. \tabularnewline
  \hline
\end{tabularx}
\end{center}
\caption{Tableau récapitulatif des tests}
\end{figure}
\fi

\chapter{Conclusion}




\section{Lien avec les connaissences obtenu lors de la formation universitaire}
METTRE DANS LA CONCLUSION
\subsection{MVC}
\subsection{Servlet}
\subsection{raeeMarker -JSP}
\subsection{Notion d'entié - Symony}
\subsection{Routage}
\subsection{Gradle Maven}
\subsection{Testes Unitaires}
L'utilité des testes : surtout démontrer l'utilisation 

\iffalse
\input{./resultats.tex}

\input{./bilan.tex}

\input{./annexes.tex}
\fi
\newpage

%récupérer les citation avec "/footnotemark"
\nocite{*}

%choix du style de la biblio
\bibliographystyle{plain}
%inclusion de la biblio
\bibliography{bibliographie.bib}
%voir wiki pour plus d'information sur la syntaxe des entrées d'une bibliographie

\end{document}