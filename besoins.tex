\chapter{Travail réalisé}


\section{Aperçu général}
\ganttset{%
	calendar week text={%
		\pgfcalendarmonthshortname{\startmonth}~\startday%
	}%
}
\newganttlinktype{f-m}{
	\ganttsetstartanchor{on right=1}
	\ganttsetendanchor{on left=0}
	\draw[/pgfgantt/link]
	([xshift=-.2pt]\xLeft, \yUpper) --       % xshift to fit arrow
	node[pos=.5, /pgfgantt/link label node] {\ganttlinklabel} 
	(\xRight, \yLower);
}


%vgrid={*1{blue!30},
%	*6{black,dotted},
%	*1{red!30},
%	*2{black,dotted},
%	*1{blue!30},
%	*{34}{black,dotted},
%	*1{green!30},
%	*1{red!30},
%	*{10}{black,dotted},
%	*1{green!30}},
\setganttlinklabel{f-m}{}

\begin{ganttchart}[
	hgrid,
	vgrid,
	x unit=3mm,
	time slot format=isodate,
	inline,
	bar/.append style={fill=blue!37},
	group/.append style={draw=black, fill=black!50},
	milestone/.append style={fill=green, rounded corners=6pt,scale=2},
	milestone inline label node/.append style={right=1mm},
	]{2019-03-28}{2019-05-25}
	\gantttitlecalendar{year, month=name, week} \\
	\ganttgroup{Analyse des besoins}{2019-03-29}{2019-04-7}\\
	\ganttgroup{Réalisation technique}{2019-04-5}{2019-05-13}\\
	\ganttgroup{Maintenance}{2019-05-13}{2019-05-24} \\
	\ganttbar{OFBiz}{2019-04-01}{2019-04-11}\\
	\ganttbar[
	bar/.append style={ fill=red!50
	}]{REST}{2019-04-08}{2019-04-28} \\
	\ganttbar[
	bar/.append style={ fill=orange!50
	}]{Entitymaint}{2019-04-20}{2019-05-12} \\
	
	\ganttmilestone{Preuve de concept}{2019-05-12}] \\
	\ganttbar[
	bar/.append style={ fill=purple!40, dashed
	}]{Revue de code}{2019-05-13}{2019-05-23} \\
	\ganttlink{elem3}{elem4}
	\ganttlink{elem4}{elem5}
	\ganttlink[link type=f-m]{elem5}{elem6}
	\ganttlink[link type=dr]{elem4}{elem6}
	\ganttlink[link type=f-m]{elem6}{elem7}
\end{ganttchart}









\section{Environnement}

\subsection{Installation du l'environnement}

Avant tout, mon intégration dans l'entreprise a commencé par l'installation du  poste de travail  
\subsection{Conventions}
\subsection{Formation développeur générale}
\subsection{Jira}
\subsection{Approfondissement de Git }
\subsection{Découverte de communauté libre Apache}

\section{Prise en main d'OFBiz}

\subsection{Premier plugin}

\subsection{Projets existants et leur structure}
\subsubsection{Decathlon}
RFID et tout ça
\subsubsection{Dejbox}
Pierre et Antoine ont tout géré 

\subsection{Problématique vis-à-vis du développement}
What is "fonctionnel"

\section{Analyse de l'existant}
\subsection{ControlServlet}
\subsection{Mécanisme de résolution des URI}
\subsection{Filtres}
Delegateur et Dispatcher


\section{Analyse des besoins et attentes de la maîtrise d'ouvrage}
\subsection{Structure générale des application web}
Les enjeux, les problématiques les solutions, *COURS MAURIZIO*
\subsection{API en cours}
RPC
\subsection{Controleur}
<request-map>...
\subsection{Besoins d'évolution}
Avenir
*Discussion communautaire*
\subsection{Representational state transfer}
\subsubsection{Histoire}
Roy Fielding
\subsubsection{Principe}
*Détailles du cours de Maurizio: idempotence, navigabilité par hyperlink, 
notion de ressource etc.
\subsubsection{Avantages}
\subsubsection{Examples d'API du style REST}
API REST de Twitter, SoundCloud, Wiktionnaire,\\
les différences entre la définition de Roy Fielding et l'implémentation de ces dernières

\subsection{Implementations existantes}
\subsubsection{Camel}
\subsubsection{JAX-RS}
Tentative d'intégration ---\\
ServletJaxRS fonctionnelle\\
Particularités techniques (annotations) \\
Conflit politique car n'est pas dans le même esprit de l'existant.\\





\section{Réalisations techniques}

\subsection{Librairie CXF}
Problèmatique avec les dépendances supplementaires: 
Tika contient déjà le CXF
\subsection{Choix vers URITemplate}
description de classe
\subsection{\textit{OverrideView()} et le conflit avec les URI segmentées}
\subsection{Choix d'intégration en parallèle avec le système existant }
\subsection{Nouveau contrôleur}
\subsubsection{Compromis pour les conflits d'URI}

\subsection{Modification de la partie "Administration: gestion des entités"  (entitymaint)  }
\subsubsection{Choix de la partie illustrative}
\subsubsection{PUT vs POST}
\subsubsection{Clés composées}
\subsubsection{Formulaires génériques }
Create update dans un même formulaire.
\subsection{Stateless}
\subsubsection{Les réalisation par la communauté}

Jaques Le Roux Token en gardant la session.
\subsection{RESTClient pour la communauté}
\subsubsection{Généralisation de code}
\subsubsection{Correction d'incohérences}

