\chapter{Travail réalisé}

\section{Environnement}

\subsection{Installation du l'environnement}
\subsection{Conventions}
\subsection{Formation développeur générale}
\subsection{Jira}
\subsection{Approfondissement de Git }
\subsection{Découverte de communauté libre Apache}

\section{Prise en main d'OFBiz}

\subsection{Premier plugin}

\subsection{Projets existants et leur structure}
\subsubsection{Décathlon}
RFID et tout ça
\subsubsection{Dejbox}
Pierre et Antoine ont tout géré 

\subsection{Problématique vis-à-vis du développement}
What is "fonctionnel"

\section{Analyse de l'existant}
\subsection{ControlServlet}
\subsection{Mécanisme de résolution des URI}
\subsection{Filtres}
Delegateur et Dispatcher


\section{Analyse des besoins et attentes de la maîtrise d'ouvrage}
\subsection{Structure générale des application web}
Les enjeux, les problématiques les solutions, *COURS MAURIZIO*
\subsection{API en cours}
RPC
\subsection{Controleur}
<request-map>...
\subsection{Besoins d'évolution}
Avenir
*Discussion communautaire*
\subsection{Representational state transfer}
\subsubsection{Histoire}
Roy Fielding
\subsubsection{Principe}
*Détailles du cours de Maurizio: idempotence, navigabilité par hyperlink, 
notion de ressource etc.
\subsubsection{Avantages}
\subsubsection{Examples d'API du style REST}
API REST de Twitter, SoundCloud, Wiktionnaire,\\
les différences entre la définition de Roy Fielding et l'implémentation de ces dernières

\subsection{Implementations existantes}
\subsubsection{Camel}
\subsubsection{JAX-RS}
Tentative d'intégration ---\\
ServletJaxRS fonctionnelle\\
Particularités techniques (annotations) \\
Conflit politique car n'est pas dans le même esprit de l'existant.\\





\section{Réalisations techniques}

\subsection{Librairie CXF}
Problèmatique avec les dépendances supplementaires: 
Tika contient déjà le CXF
\subsection{Choix vers URITemplate}
description de classe
\subsection{\textit{OverrideView()} et le conflit avec les URI segmentées}
\subsection{Choix d'intégration en parallèle avec le système existant }
\subsection{Nouveau contrôleur}
\subsubsection{Compromis pour les conflits d'URI}

\subsection{Modification de la partie "Administration: gestion des entités"  (entitymaint)  }
\subsubsection{Choix de la partie illustrative}
\subsubsection{PUT vs POST}
\subsubsection{Clés composées}
\subsubsection{Formulaires génériques }
Create update dans un même formulaire.
\subsection{Stateless}
\subsubsection{Les réalisation par la communauté}

Jaques Le Roux Token en gardant la session.
\subsection{RESTClient pour la communauté}
\subsubsection{Généralisation de code}
\subsubsection{Correction d'incohérences}



\iffalse
\section{Besoins fonctionnels}

Après une analyse des besoins fonctionnels du projet, nous avons défini deux sous catégories. D'un côté, les besoins [...], de l'autre, les besoins [...].

\subsection{Sous-partie 1}

Bla

\subsection{Sous-partie 2}

Bla

\newpage

\section{Besoins non-fonctionnels}

Comme précédemment, nous avons choisi de distinguer deux catégories pour les besoins non-fonctionnels. D'une part, nous avons les besoins non-fonctionnels pour les [...], et d'autre part ceux pour [...]. Nous avons aussi pris en compte les contraintes de développement, que nous détaillerons à la fin de cette partie.

\subsection{Sous-partie 1}

Bla\\

Aperçu du rendu souhaité :

\begin{figure}[!h]
\begin{center}
\includegraphics[height=10cm]{besoins/rendu}
\end{center}
\caption{Rendu attendu}
\end{figure}

\subsection{Sous-partie 2}

Bla

\newpage

\section{Développement}

Intro

\subsection{Tâches}

Bla\\


%tableau à taille fixée sur certaines colonnes (param sur la ligne \begin{tabularx}, voir wiki pour plus d'info sur la syntaxe
\begin{figure}[!h]
\begin{center}
\begin{tabularx}{17cm}{|c|p{6cm}|X|}
  \hline
  Priorité & Nom & Raison\\
  \hline
  1 & Tache 1 & Doit être vérifié en premier car sinon [...] \tabularnewline
  2 & Tache 2 & On doit pouvoir [...] \tabularnewline
  3 & Tache 3 & Comme les principales fonctionnalités permettant de tester sont opérationnelles, nous pouvons passer à cette tâche. \tabularnewline
  4 & Tache 4 & Parce que [...] \tabularnewline
  5 & Tache 5 & La tache 5 fait partie des principales [...]. \tabularnewline
  6 & Tache 6 & Dernière fonctionnalité essentielle à mettre en place. \tabularnewline
  7 & Tache 7 & Non-essentiel, mais apporterait un plus au projet. \tabularnewline
  8 & Tache 8 & Non-essentiel, mais apporterait un plus au projet. \tabularnewline
  \hline
\end{tabularx}
\end{center}
\caption{Tableau récapitulatif des tâches}
\end{figure}

\subsection{Tests}

Bla\\

\begin{figure}[!h]
\begin{center}
\begin{tabularx}{17cm}{|p{6cm}|X|}
  \hline
  Fonctionnalité & Test\\
  \hline
  Fonction 1 & Quand [...], vérifier [...]. \tabularnewline
  & Et quand [...], vérifier [...]. \tabularnewline
  Fonction 2 & Vérifier [...]. \tabularnewline
  Fonction 3 & Vérifier [...]. \tabularnewline
  Fonction 4 & Avoir [...]. \tabularnewline
  Fonction 5 & Accéder à [...]. \tabularnewline
   & Vérifier que [...]. \tabularnewline
  Fonction 6 & Accéder à [...]. \tabularnewline
   & Et vérifier [...]. \tabularnewline
  Fonction 7 & Installer [...]. \tabularnewline
   & Vérifier [...]. \tabularnewline
  Fonction 8 & Compter [...]. \tabularnewline
  \hline
\end{tabularx}
\end{center}
\caption{Tableau récapitulatif des tests}
\end{figure}
\fi