\chapter{Conclusion}
Ce stage a été une excellente opportunité pour mettre en œuvre mes compétences techniques acquises lors de ma formation universitaire. Malgré les difficultés rencontrées vis-a-vis de la partie fonctionnelle du travail, j'ai pu me focaliser sur l'aspect purement technique ce qui m'a été très formateur. 

C'est pour la première fois que j'ai été confronté à travailler dans le carde d'un aussi vaste projet. J'ai beaucoup apprécié la possibilité de voir l'organisation du travail dans une entreprise, notamment à travers l'utilisation avancée des outils de versionnage ainsi que des méthodes agiles. Quant à l'aspect humain, le fait de faire mon stage dans une \emph{SCOP} m'a permis de découvrir ce mode de fonctionnement d'entreprise qui m'était inconnu auparavant. Cette atmosphère démocratique et égalitaire m'a aidé à définir mes envies vis-à-vis de ma future vie professionnelle.

Mes contributions en matière d'adaptation du framework technique ont prouvé leur utilité dans un des projets clients, ainsi, je considère mes objectifs personnels et le contrat pleinement remplis. Les modification de code que j'ai proposé, sont en train d'être validés par la communauté et n'ont pas encore été appliqué dans le projet principal.
Pour l'avenir, vu les avantages de nouveau système, les utilisateurs de OFBiz ont l'intérêt d'adapter les API nouvelles et existantes suivant ce style architectural.



\section{Lien avec les connaissences obtenu lors de la formation universitaire}
Étant en Ingénierie Informatique, je ne maîtrise pas certains aspects fonctionnels incontournables au développent des projets clients. Malgré cela, j'ai pu facilement faire le lien entre les concepts techniques utilisées dans le framework et ceux que j'ai étudié à l'université. 
\subsection{MVC}
OFBiz est un framework qui se base sur une architecture MVC, étudié dans les modules comme \emph{Framework Web}, \emph{Conception orienté objet} et bien d'autres. 
\subsection{Servlet}
La notion des servlettes, vue dans le module de \emph{Programmation N-tiers} traite en profondeur le fonctionnement de ces dernières. Ces connaissances m'ont été extrêmement utiles lors de l'étude des servlettes implémentées par le framwork utilisé.   
\subsection{FreeMarker -JSP}
Le moteur de "template" vu également dans le module de \emph{Programmation N-tiers} est très similaire à celui que j'ai découvert dans le framework OFBiz. Les syntaxes quasi similaires de JSP et de Apache FreeMarker m'ont permis de vite prendre en main cet outil particulier. 
\subsection{Routage}
Les notions de routage vues dans le même module, ainsi que dans la matière \emph{Framework web}, ont facilité la compréhension de la partie principale de mon stage. 
\subsection{Notion d'entié - Symfony}
Techniquement parlant, les entités sont des objets qui correspondent aux tables des bases de données. Ces objets sont au cœur de OFBiz. C'est grâce à la découverte du framework Symfony que cette notion m'était connue. 
\subsection{Gradle - Maven}
Quant à la gestion des dépendances, les connaissances de Maven ont facilité le basculement vers un outil similaire - Gradle. 
\subsection{Testes Unitaires}
Finalement, j'ai écrit des testes unitaires JUnit pour des nombreux projets universitaires. Ainsi, j'ai pu approfondir ma compréhension de ces outils grâce au framework Mockito. J'ai découvert notamment que l'utilité de ces testes est non seulement de garder l'intégrité et la cohérence du système après les modifications, mais aussi de donner une vision générale sur le principe de fonctionnement des changements qui viennent d'être introduites. 









