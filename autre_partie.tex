\chapter{Conclusion}
Ce stage a été une excellente opportunité pour mettre en œuvre mes compétences techniques acquises lors de ma formation universitaire. Malgré les difficultés rencontrées vis-a-vis de la partie fonctionnelle du travail, j'ai pu me focaliser sur l'aspect purement technique ce qui m'a été très formateur. 

C'est pour la première fois que j'ai été confronté à travailler dans le carde d'un aussi vaste projet. J'ai beaucoup apprécié la possibilité de voir l'organisation du travail dans une entreprise, notamment à travers l'utilisation avancée des outils de versionnage ainsi que des méthodes agiles. Quant à l'aspect humain, le fait de faire mon stage dans une \emph{SCOP} m'a permis de découvrir ce mode de fonctionnement d'entreprise qui m'était inconnu auparavant. Cette atmosphère démocratique et égalitaire m'a aidé à définir mes envies vis-à-vis de ma future vie professionnelle.

Mes contributions en matière d'adaptation du framework technique ont prouvé leur utilité dans un des projets clients, ainsi, je considère mes objectifs personnels et le contrat pleinement remplis. Les modification de code que j'ai proposé, sont en train d'être validés par la communauté et n'ont pas encore été appliqué dans le projet principal.
Pour l'avenir, vu les avantages de nouveau système, les utilisateurs de OFBiz ont l'intérêt d'adapter les API nouvelles et existantes suivant ce style architectural.

***La cohérence du stage vis à vis du niveau et du contenu de la formation
universitaire ?
***Les impressions diverses, les suggestions ?


\section{Lien avec les connaissences obtenu lors de la formation universitaire}
METTRE DANS LA CONCLUSION
\subsection{MVC}
\subsection{Servlet}
\subsection{raeeMarker -JSP}
\subsection{Notion d'entié - Symony}
\subsection{Routage}
\subsection{Gradle Maven}
\subsection{Testes Unitaires}
L'utilité des testes : surtout démontrer l'utilisation 
