\chapter{Contexte du stage}

Intro

\section{Entreprise}
\subsection{Presentation générale }
SCOP Situé à Tours...

\subsection{Domaine}
Retail
\subsection{Activité}
Dévéloppement spécifique 
Maintenance et support applicatif
Administration système
\subsection{Projets}

\section{L'architecture de OFBiz}
\subsection{Vue d'ensemble sur l'architecture}
DSL en XML, servlettes, jdbc ...
\subsection{Composants }

\subsection{Container}
\subsection{Web applications}
\subsection{Entity engine}
\subsection{Service engine}
\subsection{Screen engine}
\subsection{Fonctionnel métier}
\subsection{Lien avec les connaissences obtenu lors de la formation universitaire}
METTRE DANS LA CONCLUSION
\subsubsection{MVC}
\subsubsection{Servlet}
\subsubsection{FreeMarker -JSP}
\subsubsection{Notion d'entié - Symfony}
\subsubsection{Routage}

\section{Sujet de stage }



\subsection{API REST au sein d'OFBiz}
\iffalse
Bla


Voici un tableau (cf. fig. 2.1) récapitulatif de notre analyse de l'existant...\\

%tableau centré à taille variable qui s'ajuste automatiquement suivant la longueur du contenu
\begin{figure}[!h]
\begin{center}
\begin{tabular}{|l|l|l|l|l|}
  \hline
  Solution & Critère 1 & Critère 2 & Critère 3 & Critère 4\\
  \hline
  Solution 1(cf. ref. \cite{cite0}) & Oui & Oui & Oui & Oui \\
  Solution 2(cf. ref. \cite{cite1}) & Oui & Oui & Oui & Non \\
  Solution 3(cf. ref. \cite{cite2}) & Oui (sauf telle chose) & Non & Non & Oui\\
  Solution 4(cf. ref. \cite{cite3}) & Oui& Non & Oui & Non\\
  Solution 5(cf. ref. \cite{cite4}) & Oui (uniquement ceux-ci) & Non & Oui & Non\\
  \hline
\end{tabular}
\end{center}
\caption{Tableau récapitulatif des solutions}
\end{figure}
\fi
