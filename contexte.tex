\chapter{Contexte du stage}

Intro

\section{Entreprise}

\subsection{Presentation générale }
Néréide est une société de services en logiciels libres crée en 2004  SCOP SARL \\
parler de méthodes agiles. 
transparence interne totale ;
fonctionnement démocratique (SCOP, co-gérance tournante);
implication des salariés dans tous les domaines de décision ;
salaire unique


























\subsection{Domaine}
Retail
\subsection{Activité}
\label{activite}
Transparence libre entreprise \\


Integration(utilisation des briques en tant que telles) et
Dévéloppement spécifique (adaptation d'OFBIZ pour les besoins ) plugins propres à la logique métier. Pas de forfait(contrat doit etre rempli, l'entreprise s'engage à ce qu'il soit livré dans les tepms), mais la régie (sont payé au temps de travail et pas à condition de remplir un ... ).  Pareil avec Décath 6 pers font de la régie...
Administration système
\subsection{Projets}
Décathlon
 rfid store, interface pour les achteurs et producteurs, dev front pour le store specifique au lieu d'écrans OFBiz, communication avec des API...

Dejbox
\\
Dejbox est une société de la foodTech qui propose aux salariés d’entreprise de leur livrer des repas directement sur leur lieu de travail. L’ensemble des vente est réalisé au travers d’un site e-commerce par lequel le salarié commande un repas.

Le projet a pour objectif de mettre en œuvre un outil de type ERP afin de gérer la chaîne de réapprovisionnement en produit frais vendu en ligne. Il s’agit donc de créer un référentiel d’article et de fournisseur et de pouvoir saisir des commandes qui seront envoyées aux fournisseurs et réceptionnées suite à leur livraison. Enfin, il s’agit de mettre en place la sortie de stock en intégrant les consommations de produit provenant du site de vente en ligne.


....
Dépuis 2013 les dev remontantes à la communauté , parce que besoin de support,  divergence. moutons-acteur 






\newpage
\section{Framework OFBiz}
\subsection{Vue d'ensemble }
\emph{Open For Business (OFBiz)} est une suite d'applications pour la gestion de l'entreprise qui se base sur une architecture très couramment utilisé \emph{(MVC)} et qui implémente des composantes classiques de gestion des donnés, de logique métier, et de traitement spécialisé. 

On peut notamment distinguer les modules génériques destinées à la gestion des tâches communes à la plupart des entreprises, telles que la gestion des stocks, la comptabilité, la facturation et bien d'autres. Quant à leur structure, toutes les composantes sont étroitement liées entres elles, ce qui facilite la compréhension, l'utilisation et la personnalisation de ces dernières. 


En plus d'une architecture qui encourage la customisation, OFBiz est entièrement distribué en tant que \emph{open source software}\footnote{Logiciel libre sous licence \href{https://www.apache.org/licenses/LICENSE-2.0.html}{ASL2 (Apache License Version 2.0)} ce qui donne le droit de personnaliser, d'étendre, de restructurer et de vendre le système concerné. } ce qui le rend particulièrement intéressant car le logiciel développé à base d'OFBiz n'est pas soumis à la condition d'être libre comme c'est le cas de la licence GPL  \footnote{\href{http://www.gnu.org/licenses/gpl-3.0.html}{GNU General Public Licence}} par exemple.

\subsection{Architecture }
D'un point de vue purement technique OFBiz se base sur la plateforme Java ainsi que sur l'utilisation des DSL\footnote{Domaine specific language \emph{(Langages spé-cifiques au domaine)}} basés sur des grammaires écrites en XML (mettre la bib). En ce qui concerne la partie principale du framework, les échanges HTTP sont implémentes par une extension de la classe \verb=HTTPServlet= \cite{chan2017servlet} et la communication avec les bases de données se fait via l'API Java JDBC \cite{JDBC}.

Dans sa structure on distingue \emph{le framework}, \emph{les applications} et \emph{les plugins}. Le \emph{framework} comporte l'ensemble des outils et des mécanismes techniques, notamment en matière de  communication réseau et d'interaction entre les différentes applications.
Les principaux composants métier tels que la comptabilité, la gestion des stock, ou la facturation se trouvent dans la partie \emph{applications}. 
Finalement la notion du plugin correspond à une application spécifique qui repose sur des composantes générales: par exemple le plugin \emph{eCommerce} correspond à un boutique en ligne qui utilise des nombreuses  \emph{applications} comme \emph{la gestion du stock} ou \emph{la facturation}. 

\subsection{DSL XML}
L'une des particularités d'OFBiz ce sont des fichiers XML qui servent à déclarer entre autres
des routes HTTP, des pages de rendu appelés \emph{Écrans}, ainsi que des services. Le principe est de transformer des informations sous format XML facilement compréhensibles par le développeur, en objets Java correspondants. 


\definecolor{dollarbill}{rgb}{0.52, 0.73, 0.4}
\subsection{Container}
L'interface container représenté sur la figure \ref{container} permet de définir des objets qui correspondent à des processus qui peuvent être initialisés, démarrés et arrêtés. L'intérêt est de pouvoir lancer un daemon spécifique en parallèle de l'execution d'OFBiz comme c'est le cas de \verb|EntityDataLoadContainer| qui est responsable du chargement des données et leur mise à jour en cas de modification du modèle. Quand à \verb|TestRunContainer| il s'assure du lancement des testes unitaires grâce à un mécanisme spécifique du framework. 
\begin{figure}
	\centering
	\begin{tikzpicture}
	\umlinterface[scale=0.9,x=0,y=5]{Container}{}{
		\umlvirt{+ init(cmds : List<StartupCommand>, name : String,} \\
		\umlvirt{\hspace{1,1cm}config : String) : void} \\
		\umlvirt{+ start() : void} \\
		\umlvirt{+ stop() : void} \\
		\umlvirt{+ getName() : String}
	}
	\umlclass[ x=-5, y=-2]{EntityDataLoadContainer}{
		- name : String \\
		- module : String
	}{
		\umlvirt{+ getName() : String} \\
		\umlvirt{+ createDbConstraints(...) : void}\\
		\umlvirt{+ dropPrimaryKeys(...) : void}
	}
	\umlclass[ x=6, y=-2]{TestRunContainer}{
		- name : String \\
		- jsWrapper : JunitSuiteWrapper
	}{
		\umlvirt{+ getName() : String} \\
		\umlvirt{+ createJunitXmlListener(...) : JunitXmlListener} \\
		\umlvirt{+ logTestSuiteResults(...) : void} \\
	}

 \umldep[geometry=|-|, pos1=1.5, pos2=0.2 ,draw=dollarbill,  thick]{TestRunContainer}{Container}
 \umldep[geometry=|-|, pos1=1.5, pos2=0.2 ,draw=dollarbill,  thick]{EntityDataLoadContainer}{Container}
	\end{tikzpicture}
	
	
	\caption{Définition du type container}
	\label{container}
\end{figure}
\subsection{Composants}
Les éléments constitutifs de OFBIZ sont des composants. Un composant est un regroupement des containers, des entités, des services, des vues (\emph{Écrans}) et des applications Web.


L'exemple classique d'un composant est celui de \verb|webtools| qui assure la gestion technique de l'ensemble du système par l'administrateur via une application web, ce qui implique le fait que ce composant regroupe la plupart des éléments majeurs du framework.
Nous en tant que développeurs avons la possibilité de définir nos propres composants, notamment des \emph{plugins}. 
  


\subsection{Web applications}
Des composant OFBiz ne peuvent pas être accédés directement par les utilisateurs, ils servent simplement à organiser le framework en parties individuelles de chaque aspect de l'ERP afin de faciliter leur gestion. Les applications web \emph{(webapps)} sont destinées à fournir le front-end afin que les utilisateurs puissent interagir avec OFBiz. En ce qui concerne les routes HTTP, définis classiquement dans le fichier \verb|web.xml|, dans le cas de OFBiz leur gestion est délégué à un ficher \verb|controller.xml| qui à son tour associe des traitement spécifiques à chaque point d'entrée HTTP ainsi que la valeur de retour qui peut être une vue \emph{'un Écran)}, du type \verb|JSON| ou bien une redirection. Cela se fait au moyen d'une \verb|request-map| comme on peut voir sur \ref{reqmap}



\lstset{language=XML}
\begin{figure}
\begin{lstlisting}
<request-map uri="stock" method="get">
    <event type="service" invoke="getStock"/>
    <response name="success" type="view" value="stockScreen"/>
</request-map>
\end{lstlisting}
	\caption{Association d'un point d'entrée et d'une réponse}
\label{reqmap}
\end{figure}



\subsection{Entity engine}
Comme dans beaucoup d'autres frameworks, l'interaction avec les bases de données à une place principale dans le OFBiz. Le moteur d'entités (\emph{Entity engine}) se charge de la communication avec les  bases de données à travers les déclarations uniformes, c'est à dire qui changent pas peu importe le choix de l'outil externe de gestion.



\subsection{Service engine}
Les services web assurent les échanges d'information être les applications, communément via protocole \verb| HTTP|.   
Les services OFBiz fonctionnent dans une architecture orientée service (SOA). Non seulement ces services ont une capacité d'évoquer les autres intérieurement, mais peuvent aussi être appelés par une application extérieure en utilisant des protocoles d'échange d'information telles que \verb|SOAP|. 

Les services OFBiz sont appelées en passant un contexte \footnote{Définis souvent dans les paramètres de la requête \verb|HTTP| } et retournent une réponse parmi celles conventionnellement nommés: \emph{"success"}, comme on peut voir dans  \ref{reqmap} , \emph{"error"} ou \emph{"failure"} ainsi que l'ensemble des données retournées par le service. 

On peut voir la définition d'un service sur \ref{servicedef} , qui montre notamment les attributs attendu par le services qui sont défini de deux manières: En utilisant le mécanisme de \verb|auto-attributes| qui génère des attributs en entrée (de paramètre \verb|IN|) à partir de l'ensemble des clés primaires de l'entité \verb|Stock|. L'autre manière de faire est de rajouter des attributs manuellement comme on peut le voir dans la suite de l'exemple. 


\begin{figure}
\begin{lstlisting}
<service name="getStock" engine="entity-auto" default-entity-name="Stock">
  <auto-attributes include="pk" mode="IN" optional="false"/>
  <attribute name="authKey" type="String" mode="IN" optional="true"/>
  <attribute name="stockList" type="String" mode="OUT" optional="true"/>
</service>
\end{lstlisting}
\caption{Association d'un point d'entrée et d'une réponse}
\label{servicedef}
\end{figure}


\subsection{Screen engine}
\subsection{Fonctionnel métier}

\newpage
\section{Sujet de stage }



\subsection{API REST au sein d'OFBiz}

Jira
